\documentclass{article}
\title{Projects for 2012 MRC Arithmetic Statistics Workshop}
\usepackage[margin=1.5in]{geometry}
\usepackage{hyperref}


% macros.tex
\usepackage{amsmath}
\usepackage{amsfonts}
\usepackage{amssymb}
\usepackage{amsthm}

\usepackage{url}

\newcommand{\n}{\mathfrak{n}}


% You change everything, by adding \usepackage{times} to the document
% Preamble. Now all the roman letters will be set in times and all the
% sans serif stuff will be set in Helvetica. If you don't like times,
% you can try the packages: palatcm, charter, helvet, palatino, avant,
% newcent and bookman
% If you want to change explicitly to a certain font, use the command
% \fontfamily{XYZ}\selectfont whereby XYZ can be set to: pag for Adobe
% AvantGarde, pbk for Adobe Bookman, pcr for Adobe Courier, phv for
% Adobe Helvetica, pnc for Adobe NewCenturySchoolbook, ppl for Adobe
% Palatino, ptm for Adobe Times Roman, pzc for Adobe ZapfChancery
\newcommand{\courier}{\fontfamily{pcr}\selectfont}



\newcommand{\edit}[1]{\footnote{[[#1]]}\marginpar{\hfill {\sf[[\thefootnote]]}}}
%\newcommand{\edit}[1]{{\sl\small [[Todo: #1]]}}


%\author{William~A. Stein}

\newcommand{\Hbar}{\overline{H}}

\newcommand{\myhead}[3]{
\par\noindent
{Version #2}
\vspace{10ex}
\par\noindent
{\bf \LARGE #1}\\
\vspace{3ex}
\par\noindent
{\large W.\thinspace{}A. Stein}\\
{\small Department of Mathematics, Harvard University}\vspace{1ex}\\
#3     
\vspace{2ex}\par
}

\newcommand{\myheadauth}[3]{
\par\noindent
{Version #2}
\vspace{10ex}
\par\noindent
{\bf \LARGE #1}\\
\vspace{3ex}
\par\noindent
#3     
\vspace{5ex}\par
}

\usepackage{xspace}  % to allow for text macros that don't eat space 
\newcommand{\SAGE}{{\sf Sage}\xspace}
\newcommand{\sage}{\SAGE}
\newcommand{\gzero}{\Gamma_0(N)}
\newcommand{\esM}{\overline{\sM}}
\newcommand{\sM}{\boldsymbol{\mathcal{M}}}
\newcommand{\sS}{\boldsymbol{\mathcal{S}}}
\newcommand{\sB}{\boldsymbol{\mathcal{B}}}       
\newcommand{\bA}{\mathbb{A}}
\newcommand{\cK}{\mathcal{K}}
\newcommand{\Adual}{A^{\vee}}
\newcommand{\Bdual}{B^{\vee}}
\newcommand{\kr}[2]{\left(\frac{#1}{#2}\right)}

\newcommand{\defn}[1]{{\em #1}}
\newcommand{\solution}[1]{\vspace{1em}%
  \par\noindent{\bf Solution #1.} }
\newcommand{\todo}[1]{\noindent$\bullet$ {\small \textsf{#1}} $\bullet$\\}
\newcommand{\done}[1]{\noindent {\small \textsf{Done: #1}}\\}
\newcommand{\danger}[1]{\marginpar{\small \textsl{#1}}}
\renewcommand{\div}{\mbox{\rm div}}
\DeclareMathOperator{\GCD}{GCD}
\DeclareMathOperator{\CH}{CH}
\DeclareMathOperator{\sss}{ss}
\renewcommand{\ss}{\sss}
\DeclareMathOperator{\red}{red}
\DeclareMathOperator{\sat}{sat}
\DeclareMathOperator{\xgcd}{xgcd}
\DeclareMathOperator{\Kol}{Kol}
\DeclareMathOperator{\can}{can}
\DeclareMathOperator{\Cl}{Cl}
\DeclareMathOperator{\Mod}{Mod}
\DeclareMathOperator{\chr}{char}
\DeclareMathOperator{\charpoly}{charpoly}
\DeclareMathOperator{\cris}{cris}
\DeclareMathOperator{\dR}{dR}
\DeclareMathOperator{\Fil}{Fil}
\DeclareMathOperator{\ind}{ind}
\DeclareMathOperator{\im}{im}
\DeclareMathOperator{\oo}{\infty}
\DeclareMathOperator{\abs}{abs}
\DeclareMathOperator{\lcm}{lcm}
\DeclareMathOperator{\cores}{cores}
\DeclareMathOperator{\coker}{coker}
\DeclareMathOperator{\image}{image}
\DeclareMathOperator{\prt}{part}
\DeclareMathOperator{\proj}{proj}
\DeclareMathOperator{\Br}{Br}
\DeclareMathOperator{\Ann}{Ann}
\DeclareMathOperator{\End}{End}
\DeclareMathOperator{\Tan}{Tan}
\DeclareMathOperator{\Eis}{Eis}
\newcommand{\unity}{\mathbb{1}}
\DeclareMathOperator{\Pic}{Pic}
\DeclareMathOperator{\Tate}{Tate}
\DeclareMathOperator{\Vol}{Vol}
\DeclareMathOperator{\Vis}{Vis}
\DeclareMathOperator{\Reg}{Reg}
%\DeclareMathOperator{\myRes}{Res}
%\newcommand{\Res}{\myRes}
\DeclareMathOperator{\Res}{Res}
\newcommand{\an}{{\rm an}}
\DeclareMathOperator{\rank}{rank}
\DeclareMathOperator{\Sel}{Sel}
\DeclareMathOperator{\Mat}{Mat}
\DeclareMathOperator{\BSD}{BSD}
\DeclareMathOperator{\id}{id}
\DeclareMathOperator{\dz}{dz}
%\DeclareMathOperator{\Re}{Re}
\renewcommand{\Re}{\mbox{\rm Re}}
\DeclareMathOperator{\Imm}{Im}
\renewcommand{\Im}{\Imm}
\DeclareMathOperator{\Selmer}{Selmer}
\newcommand{\pfSel}{\widehat{\Sel}}
\newcommand{\qe}{\stackrel{\mbox{\tiny ?}}{=}}
\newcommand{\isog}{\simeq}
\newcommand{\e}{\mathbf{e}}
\newcommand{\bN}{\mathbf{N}}

% ---- SHA ----
\DeclareFontEncoding{OT2}{}{} % to enable usage of cyrillic fonts
  \newcommand{\textcyr}[1]{%
    {\fontencoding{OT2}\fontfamily{wncyr}\fontseries{m}\fontshape{n}%
     \selectfont #1}}
\newcommand{\Sha}{{\mbox{\textcyr{Sh}}}}

%\font\cyr=wncyr10 scaled \magstep 1
%\font\cyr=wncyr10

%\newcommand{\Sha}{{\cyr X}}
\newcommand{\Shaan}{\Sha_{\mbox{\tiny \rm an}}}
\newcommand{\TS}{Shafarevich-Tate group}

\newcommand{\Gam}{\Gamma}
\newcommand{\X}{\mathcal{X}}
\newcommand{\cH}{\mathcal{H}}
\newcommand{\cA}{\mathcal{A}}
\newcommand{\cF}{\mathcal{F}}
\newcommand{\cG}{\mathcal{G}}
\newcommand{\cJ}{\mathcal{J}}
\newcommand{\cL}{\mathcal{L}}
\newcommand{\cV}{\mathcal{V}}
\newcommand{\cO}{\mathcal{O}}
\newcommand{\cQ}{\mathcal{Q}}
\newcommand{\cX}{\mathcal{X}}
\newcommand{\ds}{\displaystyle}
\newcommand{\M}{\mathcal{M}}
\newcommand{\E}{\mathcal{E}}
\renewcommand{\L}{\mathcal{L}}
\newcommand{\J}{\mathcal{J}}
\DeclareMathOperator{\new}{new}
\DeclareMathOperator{\Morph}{Morph}
\DeclareMathOperator{\old}{old}
\DeclareMathOperator{\Sym}{Sym}
\DeclareMathOperator{\Symb}{Symb}
%\newcommand{\Sym}{\mathcal{S}{\rm ym}}
\newcommand{\dw}{\delta(w)} 
\newcommand{\dwh}{\widehat{\delta(w)}}      
\newcommand{\dlwh}{\widehat{\delta_\l(w)}} 
\newcommand{\dash}{-\!\!\!\!-\!\!\!\!-\!\!\!\!-} 
\DeclareMathOperator{\tor}{tor}  
\newcommand{\Frobl}{\Frob_{\ell}}
\newcommand{\tE}{\tilde{E}}
\renewcommand{\l}{\ell}
\renewcommand{\t}{\tau}
\DeclareMathOperator{\cond}{cond}
\DeclareMathOperator{\Spec}{Spec}
\DeclareMathOperator{\Div}{Div}
\DeclareMathOperator{\Jac}{Jac}
\DeclareMathOperator{\res}{res}
\DeclareMathOperator{\Ker}{Ker}
\DeclareMathOperator{\Coker}{Coker}
\DeclareMathOperator{\sep}{sep}
\DeclareMathOperator{\sign}{sign}
\DeclareMathOperator{\unr}{unr}
\newcommand{\N}{\mathcal{N}}
\newcommand{\U}{\mathcal{U}}
\newcommand{\Kbar}{\overline{K}}
\newcommand{\Lbar}{\overline{L}}
\newcommand{\gammabar}{\overline{\gamma}}
\newcommand{\q}{\mathbf{q}}
%\renewcommand{\star}{\times}
\newcommand{\gM}{\mathfrak{M}}
\newcommand{\gA}{\mathfrak{A}}
\newcommand{\gP}{\mathfrak{P}}
\newcommand{\bmu}{\boldsymbol{\mu}}
\newcommand{\union}{\cup}
\newcommand{\Tl}{T_{\ell}}
\newcommand{\into}{\rightarrow}
\newcommand{\onto}{\twoheadrightarrow}%  Surjection arrow

\newcommand{\meet}{\cap}
\newcommand{\cross}{\times}
\DeclareMathOperator{\md}{mod}
\DeclareMathOperator{\toric}{toric}
\DeclareMathOperator{\tors}{tors}
\DeclareMathOperator{\Frac}{Frac}
\DeclareMathOperator{\corank}{corank}
\newcommand{\rb}{\overline{\rho}}
\newcommand{\ra}{\rightarrow}
\newcommand{\xra}[1]{\xrightarrow{#1}}
\newcommand{\hra}{\hookrightarrow}
\newcommand{\la}{\leftarrow}
\newcommand{\lra}{\longrightarrow}
\newcommand{\riso}{\xrightarrow{\sim}}
\newcommand{\da}{\downarrow}
\newcommand{\ua}{\uparrow}
\newcommand{\con}{\equiv}
\newcommand{\Gm}{\mathbb{G}_m}
\newcommand{\pni}{\par\noindent}
\newcommand{\set}[1]{\{#1\}}
\newcommand{\iv}{^{-1}}
\newcommand{\alp}{\alpha}
\newcommand{\bq}{\mathbf{q}}
\newcommand{\cpp}{{\tt C++}}
\newcommand{\tensor}{\otimes}
\newcommand{\bg}{{\tt BruceGenus}}
\newcommand{\abcd}[4]{\left(
        \begin{smallmatrix}#1&#2\\#3&#4\end{smallmatrix}\right)}
\newcommand{\mthree}[9]{\left(
        \begin{matrix}#1&#2&#3\\#4&#5&#6\\#7&#8&#9
        \end{matrix}\right)}
\newcommand{\mtwo}[4]{\left(
        \begin{matrix}#1&#2\\#3&#4
        \end{matrix}\right)}
\newcommand{\vtwo}[2]{\left(
        \begin{matrix}#1\\#2
        \end{matrix}\right)}
\newcommand{\smallmtwo}[4]{\left(
        \begin{smallmatrix}#1&#2\\#3&#4
        \end{smallmatrix}\right)}
\newcommand{\twopii}{2\pi{}i{}}  
\newcommand{\eps}{\varepsilon}
\newcommand{\vphi}{\varphi}
\newcommand{\gp}{\mathfrak{p}}
\newcommand{\W}{\mathcal{W}}
\newcommand{\oz}{\overline{z}}
\newcommand{\Zpstar}{\Zp^{\star}}
\newcommand{\Zhat}{\widehat{\Z}}
\newcommand{\Zbar}{\overline{\Z}}
\newcommand{\Zl}{\Z_{\ell}}
\newcommand{\comment}[1]{}
\newcommand{\Q}{\mathbb{Q}}
\newcommand{\QQ}{\mathbb{Q}}
\newcommand{\GQ}{G_{\Q}}
\newcommand{\R}{\mathbb{R}}
\newcommand{\RR}{\mathbb{R}}
\newcommand{\PP}{\mathbb{P}}
\newcommand{\D}{{\mathbf D}}
\newcommand{\cC}{\mathcal{C}}
\newcommand{\cD}{\mathcal{D}}
\newcommand{\cP}{\mathcal{P}}
\newcommand{\cS}{\mathcal{S}}
\newcommand{\Sbar}{\overline{S}}
\newcommand{\K}{{\mathbb K}}
\newcommand{\C}{\mathbb{C}}
\newcommand{\CC}{\mathbb{C}}
\newcommand{\Cp}{{\mathbb C}_p}
\newcommand{\Sets}{\mbox{\rm\bf Sets}}
\newcommand{\bcC}{\boldsymbol{\mathcal{C}}}
\renewcommand{\P}{\mathbb{P}}
\newcommand{\Qbar}{\overline{\Q}}
\newcommand{\QQbar}{\overline{\Q}}
\newcommand{\kbar}{\overline{k}}
\newcommand{\dual}{\bot}
\newcommand{\T}{\mathbb{T}}
\newcommand{\TT}{\mathbb{T}}
\newcommand{\calT}{\mathcal{T}}
\newcommand{\cT}{\mathcal{T}}
\newcommand{\cbT}{\mathbb{\mathcal{T}}}
\newcommand{\cU}{\mathcal{U}}
\newcommand{\Z}{\mathbb{Z}}
\newcommand{\ZZ}{\mathbb{Z}}
\newcommand{\F}{\mathbb{F}}
\newcommand{\FF}{\mathbb{F}}
\newcommand{\Fl}{\F_{\ell}}
\newcommand{\Fell}{\Fl}
\newcommand{\Flbar}{\overline{\F}_{\ell}}
\newcommand{\Flnu}{\F_{\ell^{\nu}}}
\newcommand{\Fbar}{\overline{\F}}
\newcommand{\Fpbar}{\overline{\F}_p}
\newcommand{\fbar}{\overline{f}}
\newcommand{\Qp}{\Q_p}
\newcommand{\Ql}{\Q_{\ell}}
\newcommand{\Qell}{\Q_{\ell}}
\newcommand{\Qlbar}{\overline{\Q}_{\ell}}
\newcommand{\Qlnr}{\Q_{\ell}^{\text{nr}}}
\newcommand{\Qlur}{\Q_{\ell}^{\text{ur}}}
\newcommand{\Qltm}{\Q_{\ell}^{\text{tame}}}
\newcommand{\Qv}{\Q_v}
\newcommand{\Qpbar}{\Qbar_p}
\newcommand{\Zp}{\Z_p}
\newcommand{\Fp}{\F_p}
\newcommand{\Fq}{\F_q}
\newcommand{\Fqbar}{\overline{\F}_q}
\newcommand{\Ad}{Ad}
\newcommand{\adz}{\Ad^0}
\renewcommand{\O}{\mathcal{O}}
\newcommand{\A}{\mathcal{A}}
\newcommand{\Og}{O_{\gamma}}
\newcommand{\isom}{\cong}
\newcommand{\ncisom}{\approx}   % noncanonical isomorphism
\DeclareMathOperator{\ab}{ab}
\DeclareMathOperator{\alg}{alg}
\DeclareMathOperator{\Aut}{Aut}
\DeclareMathOperator{\Frob}{Frob}
\DeclareMathOperator{\Fr}{Fr}
\DeclareMathOperator{\Ver}{Ver}
\DeclareMathOperator{\Norm}{Norm}
\DeclareMathOperator{\Ind}{Ind}
\DeclareMathOperator{\norm}{norm}
\DeclareMathOperator{\disc}{disc}
\DeclareMathOperator{\ord}{ord}
\DeclareMathOperator{\GL}{GL}
\DeclareMathOperator{\PSL}{PSL}
\DeclareMathOperator{\PGL}{PGL}
\DeclareMathOperator{\Gal}{Gal}
\DeclareMathOperator{\SL}{SL}
\DeclareMathOperator{\SO}{SO}
\DeclareMathOperator{\WC}{WC}
\newcommand{\galq}{\Gal(\Qbar/\Q)}
\newcommand{\rhobar}{\overline{\rho}}
\newcommand{\cM}{\mathcal{M}}
\newcommand{\cB}{\mathcal{B}}
\newcommand{\cE}{\mathcal{E}}
\newcommand{\cR}{\mathcal{R}}
\newcommand{\et}{\text{\rm\'et}}

\newcommand{\sltwoz}{\SL_2(\Z)}
\newcommand{\sltwo}{\SL_2}
\newcommand{\gltwoz}{\GL_2(\Z)}
\newcommand{\mtwoz}{M_2(\Z)}
\newcommand{\gltwoq}{\GL_2(\Q)}
\newcommand{\gltwo}{\GL_2}
\newcommand{\gln}{\GL_n}
\newcommand{\psltwoz}{\PSL_2(\Z)}
\newcommand{\psltwo}{\PSL_2}
\newcommand{\h}{\mathfrak{h}}
\renewcommand{\a}{\mathfrak{a}}
\newcommand{\p}{\mathfrak{p}}
\newcommand{\m}{\mathfrak{m}}
\newcommand{\trho}{\tilde{\rho}}
\newcommand{\rhol}{\rho_{\ell}}
\newcommand{\rhoss}{\rho^{\text{ss}}}
\DeclareMathOperator{\tr}{tr}
\DeclareMathOperator{\order}{order}
\DeclareMathOperator{\ur}{ur}
\DeclareMathOperator{\Tr}{Tr}
\DeclareMathOperator{\Hom}{Hom}
\DeclareMathOperator{\Mor}{Mor}
\DeclareMathOperator{\HH}{H}
\renewcommand{\H}{\HH}
\DeclareMathOperator{\Ext}{Ext}
\DeclareMathOperator{\Tor}{Tor}
\newcommand{\smallzero}{\left(\begin{smallmatrix}0&0\\0&0
                        \end{smallmatrix}\right)}
\newcommand{\smallone}{\left(\begin{smallmatrix}1&0\\0&1
                        \end{smallmatrix}\right)}

\newcommand{\pari}{{\sc Pari}}
\newcommand{\magma}{{\sc Magma}}
\newcommand{\hecke}{{\sc Hecke}}
\newcommand{\lidia}{{\sc LiDIA}}

%%%% Theoremstyles
\theoremstyle{plain}
\newtheorem{theorem}{Theorem}[section]
\newtheorem{proposition}[theorem]{Proposition}
\newtheorem{corollary}[theorem]{Corollary}
\newtheorem{claim}[theorem]{Claim}
\newtheorem{lemma}[theorem]{Lemma}
\newtheorem{hypothesis}[theorem]{Hypothesis}
\newtheorem{conjecture}[theorem]{Conjecture}

\theoremstyle{definition}
\newtheorem{definition}[theorem]{Definition}
\newtheorem{question}[theorem]{Question}
\newtheorem{idea}[theorem]{Idea}
\newtheorem{project}[theorem]{Project}
\newtheorem{problem}[theorem]{Problem}
\newtheorem{openproblem}[theorem]{Open Problem}
\newtheorem{challenge}[theorem]{Challenge}

%\theoremstyle{remark}
\newtheorem{goal}[theorem]{Goal}
\newtheorem{remark}[theorem]{Remark}
\newtheorem{remarks}[theorem]{Remarks}
\newtheorem{example}[theorem]{Example}
\newtheorem{exercise}[theorem]{Exercise}

\numberwithin{equation}{section}
\numberwithin{figure}{section}
\numberwithin{table}{section}


% bulleted list environment
\newenvironment{bulletlist}
   {
      \begin{list}
         {$\bullet$}
         {
            \setlength{\itemsep}{.5ex}
            \setlength{\parsep}{0ex}
            \setlength{\parskip}{0ex}
            \setlength{\topsep}{.5ex}
         }
   }
   {
      \end{list}
   }
%end newenvironment

% bulleted list environment
\newenvironment{dashlist}
   {
      \begin{list}
         {---}
         {
            \setlength{\itemsep}{.5ex}
            \setlength{\parsep}{0ex}
            \setlength{\parskip}{0ex}
            \setlength{\topsep}{.5ex}
         }
   }
   {
      \end{list}
   }
%end newenvironment

% numbered list environment
\newcounter{listnum}
\newenvironment{numlist}
   {
      \begin{list}
            {{\em \thelistnum.}}{
            \usecounter{listnum}
            \setlength{\itemsep}{.5ex}
            \setlength{\parsep}{0ex}
            \setlength{\parskip}{0ex}
            \setlength{\topsep}{.5ex}
         }
   }
   {
      \end{list}
   }
%end newenvironment

\newcommand{\hd}[1]{\vspace{1ex}\noindent{\bf #1} }
\newcommand{\nf}[1]{\underline{#1}} 
\newcommand{\cbar}{\overline{c}}

\DeclareMathOperator{\rad}{rad}

\theoremstyle{definition}
\newtheorem{algor}[theorem]{Algorithm}
\newenvironment{algorithm}[1]{%
\begin{algor}[#1]\index{{\bf Algorithm}!#1}
}%
{\end{algor}}

\newenvironment{steps}%
{\begin{enumerate}\setlength{\itemsep}{0.1ex}}{\end{enumerate}}

\usepackage{color}
\usepackage{cprotect}
\usepackage{listings} 
\lstdefinelanguage{Sage}[]{Python}
{morekeywords={True,False,sage,singular},
sensitive=true}
\lstset{
  showtabs=False,
  showspaces=False,
  showstringspaces=False,
  commentstyle={\ttfamily\color{dredcolor}},
  keywordstyle={\ttfamily\color{dbluecolor}\bfseries},
  stringstyle ={\ttfamily\color{dgraycolor}\bfseries},
  language = Sage,
  basicstyle={\small \ttfamily},
  aboveskip=1em,
  belowskip=1em,
  backgroundcolor=\color{lightyellow},
  frame=single
}
\definecolor{lightyellow}{rgb}{1,1,.86}
\definecolor{dblackcolor}{rgb}{0.0,0.0,0.0}
\definecolor{dbluecolor}{rgb}{.01,.02,0.7}
\definecolor{dredcolor}{rgb}{0.8,0,0}
\definecolor{dgraycolor}{rgb}{0.30,0.3,0.30}
\definecolor{graycolor}{rgb}{0.35,0.35,0.35}
\newcommand{\dblue}{\color{dbluecolor}\bf}
\newcommand{\dred}{\color{dredcolor}\bf}
\newcommand{\dblack}{\color{dblackcolor}\bf}
\newcommand{\gray}{\color{graycolor}}

\newcommand{\dbd}[1]{\langle#1\rangle}   % make a diamond bracket d symbol

%%% Local Variables: 
%%% mode: latex
%%% TeX-master: t
%%% End: 







\author{William Stein}
\begin{document}
\maketitle
\begin{abstract}
  We give mathematical descriptions of the projects that we will
  attack during the Snowbird workshop.  Two projects involve classical
  modular forms, and two involve Hilbert modular forms for the field
  $\Q(\sqrt{5})$. There is more information about resources for
  sutdying these problems in the wiki
  \url{https://github.com/williamstein/mrc-2012/wiki}.
As far as I know these are unsolved research level questions. 
\end{abstract}

\tableofcontents

\section{Background}


\section{Questions about classical modular forms}

\subsection{The ordinary defect}

This is a question of Barry Mazur about classical modular forms and
Hida theory.  It involves gathering data about a question that Barry
asked me a week ago, which via some arguments I reduced to a question
about classical modular forms of level one.

For any prime number $p$ and even integer weight $k>2$, let $d(p,k)$
be the number of $p$-adic non-unit eigenvalues of the Hecke operator
$T_p$ acting on the space $S_k(\SL_2(\Z))$ of weight $k$ level $1$
classical modular forms. We call $d(p,k)$ the {\em ordinary defect},
since it measures the failure of all newforms to be ordinary.  For
theoretical reasons (Hida theory and symmetry) we are only
interested in $k\leq (p-3)/2$.  The following is a table of all such
$(p,k)$ with $d(p,k)>0$ for $p<389$:

\begin{center}
\begin{tabular}{|l|c|}\hline
$(p,k)$&{\bf defect}\\\hline
(59,16)&1\\
(79,38)&1\\
(107,28)&1\\
(131,40)&1\\
(139,36)&1\\
(151,60)&1\\
(173,24)&1\\
(193,72)&1\\
(223,72)&1\\
(229,116)&2\\
(257,50)&1\\
\hline\end{tabular}
\begin{tabular}{|l|c|}\hline
$(p,k)$&{\bf defect}\\\hline
(257,100)&1\\
(257,130)&2\\
(263,98)&1\\
(269,78)&1\\
(277,92)&1\\
(283,72)&2\\
(307,78)&1\\
(313,114)&1\\
(331,84)&2\\
(353,76)&2\\
(379,56)&1\\
\hline\end{tabular}
\end{center}

\vspace{1em}
\noindent{\bf The Challenge:} find a way to extend this table for all $p<1000$.
This will provide data for a new conjecture that Barry Mazur intends to make.
\vspace{1em}



\begin{remark} Perhaps we could do a similar computation, but with level
$1$ replaced by level $N$, where maybe we should only consider
primes $p\nmid N$.  
\end{remark}

%Include other levels?
% goal: 1000; new conjectures

%local rep on local galois barsotti tate group modular curve $X_1(p)$....
%action of inertia group on neron fiber...


\subsection{Congruences and classical higher-weight modular forms}

\subsubsection{Background: the congruence module and number}\label{sec:cong}
Let $\T \subset \End(S)$ be a Hecke algebra, where $S$ is some space
of cusp forms.  Suppose $M$ is a $\T$-module that is also of finite
rank as a $\Z$-module, and $C\subset M$ is a {\em saturated}
submodule, so $M/C$ is torsion free.  Moreover, suppose that $M_\Q =
M\tensor\Q$ splits as a direct sum $C_{\Q} \oplus C'_{\Q}$, where
$C'_{\Q}$ a $\T$-submodule.  Define the the {\em congruence module} of
$C$ to be the $\T$-module $M/(C+C')$ and the {\em congruence number}
of $C$ to be $\#(M/(C +C')) \in \Z$.  

\subsection{The Problem}
% CM forms at end of Hida 1981 paper.  Computed a little, enough to show interesting.

Consider $S=S_k(\Gamma_0(N))$ for some $N,k$, and $S(\Z)=S\cap
\Z[[q]]$.  For each of newform $f\in S$ (up to Galois conjugation) of
level dividing $N$, let $C_f \subset S$ be the $\T$-submodule of $S$
spanned by the images in $S$ via degeneracy maps of all Galois
conjugates of $f$.  Define the congruence module and congruence number
of $f$ as in Section~\ref{sec:cong} to be the corresponding values for
$C_f$ as a submodule of $S(\Z)$.  Denote by $c_f$ the congruence
number of $f$.

Given two newforms $f,g \in S$ as above such that $C_f \neq C_g$, define
the congruence module and congruence number of this pair to be
the congruence module and number of $C_f$ in the $\T$-module  
$S(\Z)\cap((C_f + C_g)\tensor\Q)$. This is a number that measures 
congruences between $f$ and $g$.

For any $N$ and $k\geq 2$, let $\cS = \cS_k(\Gamma_0(N))$ be the
complex vector space of cuspidal modular symbols of weight $k$ and
level $N$, and let $S(\Z) = \cS_k(\Gamma_0(N);\Z)$ be the submodule of
integral cuspidal modular symbols.  There is a perfect $\T$-invariant
duality $\cS \times (S \oplus \overline{S}) \to \C$ given by
integration, hence for each newform $f$ and module $C_f$ as above,
there is a corresponding submodule $D_f \subset S(\Z)$ (note that
$\rank_{\Z} D_f = 2\rank_{\Z} C_f$).  Denote by $d_f$ the congruence
number of this submodule.

In the case when $f$ is new of level $N$, has weight $2$, and has
rational Fourier coefficients (so $f$ corresponds to an elliptic curve
over $\Q$), I think $\sqrt{d_f}$ is the usual modular degree of the
optimal elliptic curve attached to $f$.  In this case, Ken Ribet
proved that $\sqrt{d_f} \mid c_f$, and Agashe, Ribet and I formulate
and prove a generalization of this theorem to $f$ with nonrational
Fourier coefficients in
\url{http://wstein.org/papers/ars-congruence/}.

When I was a grad student it seems that I computed the first ever
example illustrating that $\sqrt{d_f} \neq c_f$, which was in
$S_2(\Gamma_0(54))$. 


\vspace{1em}
\noindent{\bf The Challenge:} Compute the numbers $c_f$ and $d_f$ for
some newforms of weight $\geq 4$.  Is there a relation between
them as is the case in weight $2$?  
\vspace{1em}


It is a theorem of Ken Ribet (that is proved in Mazur's Eisenstein
ideal paper) that if $\T$ is the Hecke algebra associated to
$S_2(\Gamma_0(N))$, then the scheme $\Spec(\T)$ is connected.  At
least when $N$ is prime, I think what this amounts to is that the
following graph is connected: the vertices correspond to Galois orbits
of newforms $f$ and the orbits of $f$ and $g$ are connected by
an edge whenever $f$ and $g$ have congruence number bigger than $1$.
Ribet's argument doesn't seem to generalize to higher weight, but I
suspect the conclusion still holds...

\vspace{1em}
\noindent{\bf The Challenge:} Compute the graph defined above for
$S_k(\Gamma_0(N))$ for $k\geq 4$ and various $N$.
\vspace{1em}


\section{Questions about Hilbert modular forms}

Let $F=\Q(\sqrt{5})$.  The {\em modularity conjecture} asserts that
the set of $L$-functions $L(E,s)$ attached to elliptic curves over $F$
of conductor an ideal $\n$ of $F$ is the same as the set $L(f,s)$ of
$L$-functions attached to rational cuspidal Hilbert modular newforms of
parallel weight $(2,2)$ and level $\n$.



\subsection{Congruences for Hilbert modular forms}

Let $S_{(2,2)}(\Gamma_0(\n))$ denote the space of Hilbert modular cusp forms of
parallel weight $(2,2)$ and level $\Gamma_0(\n)$.  Let $\T$ be the Hecke algebra
acting on $S_{(2,2)}(\Gamma_0(\n))$.   Dembele's algorithm involves a finite
rank free $\Z$-module that I'll call $X$ on which $\T$ acts. 
(This is a module with basis certain orbits of $\P^1(\cO/\n)$.)
Given a newform $f\in S_{(2,2)}(\Gamma_0(\n))$, there is a corresponding
submodule $C_f \subset X$, and we may consider the congruence module
of $f$ 
and congruence number $c_f$ of $f$ inside $X$.

\vspace{1em}
\noindent{\bf The Challenge:} Compute the number $c_f$ for some Hilbert
newforms $f \in S_{(2,2)}(\Gamma_0(\n))$.
\vspace{1em}


The definition of $c_f$ above depends on some perhaps {\em ad hoc}
algorithm.  Here is a more intrinsic definition.  The Hecke algebra
$\T$ is a finite rank $\Z$-module.  Using Dembele's algorithm we can
at least compute the algebra generated by all $T_{\p}$ with
$\p\nmid\n$.  Just as above, given a newform $f$ there is saturated
submodule $D_f$ of $\T$ associated to $f$.  Let $d_f$ be the
congruence number of $D_f$.

\vspace{1em}
\noindent{\bf The Challenge:} Compute the number $d_f$ for some
Hilbert newforms $f \in S_{(2,2)}(\Gamma_0(\n))$.  What is the relation between
$c_f$ and $d_f$?
\vspace{1em}


When $f$ has rational Fourier coefficients, there is a corresponding
isogeny class of elliptic curves, and we could also study the degrees
of maps from Shimura curves to those elliptic curves.   However, I still
don't know how to explicitly compute any of those degrees yet. 


\subsection{The first elliptic curve over $F$ of  rank 3}
This project is to find the first elliptic curve over $F=\Q(\sqrt{5})$
of rank 3, assuming all standard conjectures about elliptic curves
over $F$, i.e,. assuming modularity and the Birch and Swinnerton-Dyer
conjecture.  The current rank records are:
\begin{center}
\begin{tabular}{|l|l|l|l|}\hline
Rank & Norm(N) & Equation & Person\\\hline
0 & 31 (prime) &  $[1,a+1,a,a,0]$ &  Dembele \\
1 & 199 (prime) &  $[0,-a-1,1,a,0]$ &  Dembele \\
2 & 1831 (prime) &  $[0,-a,1,-a-1,2a+1]$ & Dembele \\
3 & 26,569$\,=163^2$ &  $[0,0,1,-2,1]$ & Elkies \\
4 & 1,209,079 (prime) & $[1, -1, 0, -8-12a, 19+30a]$ & Elkies \\
5 & 64,004,329 & $[0, -1, 1, -9-2a, 15+4a]$ & Elkies
\\\hline
\end{tabular}
\end{center}

\vspace{1em}
\noindent{\bf The Challenge:} 
Enumerate all {\em rational} newforms of norm conductor up to
$26569=163^2$ to sufficient precision so that for each we can compute
$r_f = \ord_{s=1}(L(f,s))$.  If there is an $f$ with norm conductor
$<26569$ with $\ord_{s=1}(L(f,s))=3$, find a corresponding Weierstrass
equation.
\vspace{1em}


\end{document}
